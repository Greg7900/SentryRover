
\documentclass[11pt,a4paper]{article}
\usepackage[margin=2.2cm]{geometry}
\usepackage{amsmath,amssymb}
\usepackage{siunitx}
\usepackage{hyperref}
\usepackage{physics}
\usepackage{graphicx}
\usepackage{booktabs}
\hypersetup{colorlinks=true,linkcolor=black,urlcolor=blue,citecolor=black}

\title{Dimensionnement du Rover et de la Tourelle}
\author{Projet Robot Mobile -- Synthèse des calculs}
\date{\today}

\begin{document}
\maketitle
\tableofcontents
\bigskip

\section{Hypothèses et objectifs}
\subsection{Objectifs}
\begin{itemize}
  \item \textbf{Rover}: atteindre une vitesse maximale de \SI{5}{m/s} avec des roues de \(\varnothing \SI{15}{cm}\) (rayon \(R=\SI{0.075}{m}\)).
  \item \textbf{Tourelle (pan)}: effectuer \(\SI{360}{\degree}\) en \(\SI{0.2}{s}\) avec une charge de \(\SI{10}{kg}\) sur un plateau de \(\varnothing \SI{15}{cm}\).
\end{itemize}

\subsection{Composants retenus}
\begin{itemize}
  \item \textbf{Rover}: 4 moteurs brushless à réducteur M62B90-24P-30S/G62-5.36S1, \SI{90}{W}, \(\tau=\SI{1.38}{N.m}\) chacun, vitesse en sortie \(\SI{560}{rpm}\).
  \item \textbf{Tourelle}: moteur NEMA23 iSV57T-090, \SI{90}{W}, \(\tau_{nom}=\SI{0.3}{N.m}\), \(\tau_{cr}=\SI{0.8}{N.m}\), \(\SI{3000}{rpm}\); réduction par courroie \(10{:}1\).
  \item \textbf{Masse châssis nu}: \(\sim\SI{14}{kg}\) (moteurs, drivers, batterie 6S, châssis, Jetson, Arduino, câblage).
\end{itemize}

\section{Rover}
\subsection{Cinématique de la roue}
Périmètre roue \(C = 2\pi R\). Régime requis pour une vitesse \(v\):
\begin{equation}
\mathrm{RPM}=\frac{v\cdot 60}{2\pi R}.
\end{equation}
Avec \(v=\SI{5}{m/s}\) et \(R=\SI{0.075}{m}\):
\begin{equation}
\mathrm{RPM}_5=\frac{5\cdot 60}{2\pi\cdot 0.075}\approx \boxed{\SI{637}{rpm}}.
\end{equation}

\subsection{Vitesse maximale avec moteurs \SI{560}{rpm}}
\begin{align}
\mathrm{tr/s} &= \frac{560}{60}=9.33,\\
v &= \mathrm{tr/s}\cdot C = 9.33\cdot (2\pi\cdot 0.075)\approx \boxed{\SI{4.40}{m/s}}.
\end{align}

\subsection{Force motrice et accélération}
Force par moteur (couple \(\tau_m=\SI{1.38}{N.m}\)) :
\begin{equation}
F_m=\frac{\tau_m}{R}=\frac{1.38}{0.075}\approx \SI{18.4}{N}.
\end{equation}
Force totale (4 moteurs): \(\boxed{F_{tot}\approx \SI{73.6}{N}}\).\\
Accélération avec masse \(m=\SI{15}{kg}\):
\begin{equation}
a=\frac{F_{tot}}{m}=\frac{73.6}{15}\approx \boxed{\SI{4.91}{m/s^2}}.
\end{equation}

\subsection{Charge totale admissible (rover compris)}
Pour une accélération cible \(a\), la masse totale admissible vaut:
\begin{equation}
m_{\max}=\frac{F_{tot}}{a}.
\end{equation}
En retranchant la masse du rover \(m_r\simeq\SI{14}{kg}\), on obtient la charge utile \(m_{\text{payload}}\).
\begin{center}
\begin{tabular}{lcc}
\toprule
Cible & \(m_{\max}\) (total) & \(m_{\text{payload}}\) (utile)\\
\midrule
\(a=\SI{1}{m/s^2}\) & \(\SI{73.6}{kg}\) & \(\boxed{\SI{59.6}{kg}}\)\\
\(a=\SI{3}{m/s^2}\) & \(\SI{24.5}{kg}\) & \(\boxed{\SI{10.5}{kg}}\)\\
\bottomrule
\end{tabular}
\end{center}

\subsection{Variante roues \(\varnothing\SI{16}{cm}\) (mémo)}
Avec \(R=\SI{0.08}{m}\) : \(C=2\pi R=\SI{0.503}{m}\).
\begin{align}
\mathrm{RPM}_5&=\frac{5\cdot 60}{2\pi\cdot 0.08}\approx \boxed{\SI{597}{rpm}},\\
v_{560}&=\frac{560}{60}\cdot 2\pi\cdot 0.08\approx \boxed{\SI{4.69}{m/s}},\\
F_m&=\frac{1.38}{0.08}=\SI{17.25}{N},\quad F_{tot}\approx \SI{69}{N}.\\
\text{Avec } m=\SI{15}{kg}:&\quad a\approx \SI{4.6}{m/s^2}.\\
\text{Charges utiles: }&\sim \SI{55}{kg} \text{ (}\SI{1}{m/s^2}\text{)},\ \sim \SI{9}{kg} \text{ (}\SI{3}{m/s^2}\text{)}.
\end{align}

\section{Tourelle (Pan)}
\subsection{Spécifications}
Objectif: \(\SI{360}{\degree}\) en \(\SI{0.2}{s}\) avec un plateau (disque plein) \(m=\SI{10}{kg}\), \(r=\SI{0.075}{m}\).

\subsection{Vitesse et accélération angulaires}
Vitesse angulaire requise:
\begin{align}
\omega &= \frac{2\pi}{0.2}=10\pi \approx \boxed{\SI{31.416}{rad/s}},\\
\mathrm{RPM}_{plateau}&=\omega\cdot \frac{60}{2\pi}\approx \boxed{\SI{300}{rpm}}.
\end{align}
Accélération angulaire (0\(\rightarrow\)\(\omega\) en \(\SI{0.2}{s}\)):
\begin{equation}
\alpha=\frac{\Delta \omega}{\Delta t}=\frac{31.416}{0.2}\approx \boxed{\SI{157}{rad/s^2}}.
\end{equation}

\subsection{Couple d'accélération requis}
Moment d'inertie du plateau (disque plein):
\begin{equation}
I=\frac{1}{2}mr^2=\frac{1}{2}\cdot 10 \cdot 0.075^2=\SI{0.028125}{kg\,m^2}.
\end{equation}
Couple requis:
\begin{equation}
\tau_{req}=I\alpha=0.028125\times 157 \approx \boxed{\SI{4.41}{N.m}}.
\end{equation}

\subsection{Capacité moteur (iSV57T-090) avec réduction \(10{:}1\)}
Moteur: \(\tau_{nom}=\SI{0.3}{N.m}\), \(\tau_{cr}=\SI{0.8}{N.m}\), \(n_m=\SI{3000}{rpm}\).\\
Après réduction \(10{:}1\): \(n_s=\SI{300}{rpm}\), \(\tau_{nom,s}=\SI{3.0}{N.m}\), \(\tau_{cr,s}=\SI{8.0}{N.m}\).\\[4pt]
Comparaison: \(\tau_{req}=\SI{4.41}{N.m} < \tau_{cr,s}=\SI{8.0}{N.m}\) \(\Rightarrow\) conforme pour des accélérations brèves; en régime établi, le couple nécessaire est dominé par les frottements (typiquement \(\ll \SI{3}{N.m}\)).

\subsection{Remarques sur la courroie}
Pignon GT2 10 dents (pas \(\SI{2}{mm}\)): \(D \approx \frac{Np}{\pi}\approx \SI{6.37}{mm}\) (\(r\approx\SI{3.18}{mm}\)).\\
Force tangente crête côté petit pignon:
\begin{equation}
F\approx \frac{\tau_{m,cr}}{r}=\frac{0.8}{0.00318}\approx \SI{250}{N}.
\end{equation}
Recommandations: courroie GT2 PU renforcée (largeur \(\ge\)\SI{9}{mm}), tension adéquate; pour plus de marge, profils HTD-3M/5M.

\section{Synthèse}
\begin{itemize}
  \item Rover (\(\varnothing \SI{15}{cm}\)): \(v\approx \SI{4.40}{m/s}\) à \SI{560}{rpm}; \(F_{tot}\approx \SI{73.6}{N}\); charges utiles: \(\sim\SI{59.6}{kg}\) à \(\SI{1}{m/s^2}\), \(\sim\SI{10.5}{kg}\) à \(\SI{3}{m/s^2}\).
  \item Variante \(\varnothing \SI{16}{cm}\): \(v\approx \SI{4.69}{m/s}\); \(F_{tot}\approx \SI{69}{N}\).
  \item Tourelle: besoin \(\tau\approx \SI{4.41}{N.m}\); solution \(10{:}1\) avec iSV57T-090: \(\tau_{nom,s}=\SI{3.0}{N.m}\), \(\tau_{cr,s}=\SI{8.0}{N.m}\) \(\Rightarrow\) marge suffisante en crête.
\end{itemize}

\end{document}
